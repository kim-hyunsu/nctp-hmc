\documentclass{article}
\usepackage{bbm}
% if you need to pass options to natbib, use, e.g.:
    % \PassOptionsToPackage{numbers, compress}{natbib}
% before loading neurips_2018

% ready for submission
% \usepackage{neurips_2018}

% to compile a preprint version, e.g., for submission to arXiv, add add the
% [preprint] option:
%     \usepackage[preprint]{neurips_2018}

% to compile a camera-ready version, add the [final] option, e.g.:
     \usepackage[final]{neurips_2018}

% to avoid loading the natbib package, add option nonatbib:
    % \usepackage[nonatbib]{neurips_2018}

\usepackage[utf8]{inputenc} % allow utf-8 input
\usepackage[T1]{fontenc}    % use 8-bit T1 fonts
\usepackage{hyperref}       % hyperlinks
\usepackage{url}            % simple URL typesetting
\usepackage{booktabs}       % professional-quality tables
\usepackage{amsfonts}       % blackboard math symbols
\usepackage{nicefrac}       % compact symbols for 1/2, etc.
\usepackage{microtype}      % microtypography
\usepackage{algorithm}
\usepackage{algorithmic}
\usepackage{mathtools}

\title{Non-Canonical Hamiltonian Monte Carlo Algorithm for Two Particles}

% The \author macro works with any number of authors. There are two commands
% used to separate the names and addresses of multiple authors: \And and \AND.
%
% Using \And between authors leaves it to LaTeX to determine where to break the
% lines. Using \AND forces a line break at that point. So, if LaTeX puts 3 of 4
% authors names on the first line, and the last on the second line, try using
% \AND instead of \And before the third author name.

\author{%
  Sihyun Yu\\
  Department of Mathematics\\
  KAIST\\
  Yuseong-gu 291, Daejeon \\
  \texttt{yusihyunc@kaist.ac.kr} \\
  % examples of more authors
  \And
  Hyunsoo Kim\\
  Department of Physics\\
  KAIST\\
  Yuseong-gu 291, Daejeon \\
  \texttt{khszone02@kaist.ac.kr} \\
  }

\begin{document}
% \nipsfinalcopy is no longer used

\maketitle

\begin{abstract}
Probabilistic inference is one of the important concepts in machine learning. However, many problems still exist while sampling from high dimensional distribution or multimodal distributions. Many researchers try to solve this problem using existing physical concepts. We propose a new algorithm using non-canonical Hamiltonian dynamics and copied and pasted independent probabilistic distribution.  In this algorithm, two particles interacts each other while sampling the one target distribution. This interaction might solve existing problems in probabilistic sampling since interaction can make the trace of each particle more broadly.   
 \end{abstract}
\section{Introduction}
Markov Chain Monte Carlo is a method for sampling from a probability distribution. It is used to approximate the normalization term of high dimensional integral in Bayesian inference. (J. Charles \textit{et al})., 2005)

Although the existing algorithms of Markov Chain Monte Carlo, such as Metropolis-Hastings algorithm and Gibbs sampling, are easy to implement and analyze, they takes long time to explore entire space of given probability distribution. To solve this problem, in 1987, Simon Duane, A.D. Kennedy, Brian Pendleton and Dun can Roweth invented a Markov Chain Monte Carlo algorithm using Hamiltonian dynamics, which is called Hamiltonian Monte Carlo (also known as Hybrid Monte Carlo). (S. Duane \textit{et al}., 1987)

Sampling probability distributions using the Hamiltonian Monte Carlo algorithm can be thought of like a ball rolling on a mountain that represents the probability distribution. High peaks indicate high-density regions in the probability distribution. On the contrary, low-altitude mountain ranges mean areas with low density on the probability distribution. Let's assume that we roll a ball at the high peaks. At first, the ball will not be so fast. The speed will increase as the ball goes down. If we shoot this scene in periodic time, we will get more pictures of the ball when the ball is slow. In other words, more sampling is done in high mountain peaks that represent high density on a probability distribution. This is the principle of Hamiltonian Monte Carlo.

Hamiltonian Monte Carlo needs two mechanical energies to interpret a movement of a particle. It maps the probability distribution to the potential energy of a particle. In addition, It sets arbitrary kinetic energy (mostly a quadratic form of momentum). Depending on the formula of the kinetic energy of this particle, how the particle finds out the probability distribution is determined.

We have fundamentally changed the Hamiltonian Monte Carlo algorithm. Using Hamiltonian dynamics is the same, but we exploited two particles instead of one particle. Each particle traverses different regions of the probability distribution and they can communicate with each other through a kind of electromagnetic field generated from each particle. We proved those changes do not violate the theoretical validity of the Hamiltonian Monte Carlo algorithm.

\section{Theoretical Background}
\subsection{Hamiltonian Monte Carlo}
Pulling the aforementioned physical intuition of the ball back into the probabilistic perspective, the key is to expand our original probabilistic system with the introduction of auxiliary momentum parameters, $p_{n}$, to complement each dimension of our target parameter space,
\[
q_{n} \rightarrow (q_{n}, p_{n})
\]
expanding the D-dimensional parameter space into a 2D-dimensional phase space. We can now convert the target distribution into a joint probability on phase space, which is called the canonical distribution.
\[
\pi(q,p) = \pi(p|q)\pi(q)
\]
If we want to get target distribution, $\pi(q)$, we just marginalize out the momentum, $p$.
We can write it in terms of an invariant Hamiltonian, $H(q,p)$,
\[
\pi(q,p) = e^{-H(q,p)}
\]
\[
H(q,p) = -\log\pi(q,p)
\]
It preserves invariant probabilistic structure of the phase space and the value of the Hamiltonian in phase space is called the energy in physical perspective. It decomposes into two terms,
\[
H(q,p) = -\log\pi(p | q) - \log\pi(q)
\]
\[
        =       K(p,q)    +  V(q)
\]
$K(p,q)$ is called kinetic energy and $V(q)$ is called potential energy. So, we can describe the evolution of the $q$ and $p$ over time by the Hamiltonian,. which is,
\[
\frac{dq}{dt} = +\frac{\partial H}{\partial p} = \frac{\partial K}{\partial p}
\]
\[
\frac{dp}{dt} = -\frac{\partial H}{\partial q} = -\frac{\partial K}{\partial q} - \frac{\partial V}{\partial q}
\]
To convert into discretized algorithm, we will should a symplectic integrator called \textit{leapfrog} integrator,
\begin{algorithmic}
\STATE $q_0 \leftarrow q, p_0 \leftarrow p$
\FOR {$0 \leq n < \frac{T}{\epsilon}$}
\STATE $p_{n+\frac{1}{2}} \leftarrow p_{n} - \frac{\epsilon}{2}\frac{\partial V}{\partial q}(q_{n})$
\STATE $q_{n+1} \leftarrow q_{n} + \epsilon p_{n+\frac{1}{2}}$
\STATE $p_{n+1} \leftarrow p_{n+\frac{1}{2}} - \frac{\epsilon}{2}\frac{\partial V}{\partial q}(q_{n+1})$
\ENDFOR.
\end{algorithmic}
(M. Betancourt., 2014)

\subsection{Magnetic Hamiltonian Monte Carlo}
We used canonical Hamilton equations in Hamiltonian Monte Carlo. However, if some conditions are satisfied, we can use another Hamilton equations called non-canonical Hamiltonian equation. The canonical Hamiltonian equation is able to be described as
\[
\frac{d}{dt}
\left[
\begin{array}{ccc}
q(t) \\
p(t)
\end{array}
\right]
=
\left[
\begin{array}{cclcc}
\textbf{0} & \textbf{I} \\
-\textbf{I} & \textbf{0}
\end{array}
\right]
\left[
\begin{array}{ccc}
\nabla_{q} H(q(t), p(t)) \\
\nabla_{p} H(q(t), p(t))
\end{array}
\right]
\]
\[
=
\left[
\begin{array}{cclcc}
\textbf{0} & \textbf{I} \\
-\textbf{I} & \textbf{0}
\end{array}
\right]
\left[
\begin{array}{ccc}
\nabla_{q} U(q(t)) \\
\nabla_{p} K(p(t))
\end{array}
\right]
\]
\[
=
\left[
\begin{array}{ccc}
\textbf{p}(t) \\
-\nabla_{q} U(q(t))
\end{array}
\right]
\]
when $K(p(t)) = q^2/2$. On the contrary, the non-canonical Hamiltonian equation is described as
\[
\frac{d}{dt}
\left[
\begin{array}{ccc}
q(t) \\
p(t)
\end{array}
\right]
=
\left[
\begin{array}{cclcc}
\textbf{E} & \textbf{F} \\
-\textbf{F}^{T} & \textbf{G}
\end{array}
\right]
\left[
\begin{array}{ccc}
\nabla_{q} H(q(t), p(t)) \\
\nabla_{p} H(q(t), p(t))
\end{array}
\right]
\]
To check out whether it preserves its energy(Hamiltonian) since it is a necessary condition of Hamiltonian Monte Carlo, we should compute Poisson brackets. The Poisson bracket of the non-canonical Hamilton equation is following,
$$\displaystyle\frac{dQ}{dt}
 = (\nabla_{q}Q)^T F \nabla_{p}H(p,q) - (\nabla_{p}Q)^T F^T \nabla_{q}H(p,q) + (\nabla_{q}Q)^T E \nabla_{q}H(p,q) + (\nabla_{p}Q)^T G \nabla_{p}H(p,q)$$
This value should be zero to preserve the energy when Q is Hamiltonian, $H(p,q)$, that is,
\[\displaystyle(\nabla_{q}H(p,q))^T F \nabla_{p}H(p,q) - (\nabla_{p}H(p,q))^T F^T \nabla_{q}H(p,q)\]
\[+ (\nabla_{q}H(p,q))^T E \nabla_{q}H(p,q) + (\nabla_{p}H(p,q))^T G \nabla_{p}H(p,q) = 0.\]
Then, first two terms are Poisson bracket of canonical Hamilton equation, which is zero. we denote rest of the term,
$$(\nabla_{q}H(p,q))^T E \nabla_{q}H(p,q) + (\nabla_{p}H(p,q))^T G \nabla_{p}H(p,q) = 0.$$
This means both E and G are anti-symmetric matrices. (N. Tripuraneni \textit{et al}., 2017)

\section{Results}
\subsection{Algorithmic Background}

Consider the case we sample $q$ from the distribution $\pi(q)$. From now, we consider the case as sampling the distribution $\pi(q_1 , q_2 ) = \pi(q_1)\pi (q_2)$. (Copy and paste the target distribution) we assume $q_1$ and $q_2$ has the momentum, $p_1$ and $p_2$, respectively, with $p_1$, $p_2 \sim N(0,I)$. Then the Hamiltonian is defined by $\displaystyle H(q_1, q_2, p_1, p_2)  = U(q_1) +U(q_2)+\frac{1}{2}{p_1}^T p_1 + \frac{1}{2}{p_2}^T p_2$. We observe what happens to this assumption in non-canonical Hamiltonian dynamics with $A \in \mathbbm{R}^{4n \times 4n} $. We denote $A$ as
\[
A=
\left[
\begin{array}{c|c}
E & F \\
\hline
-F^{T} & G
\end{array}
\right]
\] where

\[
E=
\left[
\begin{array}{c|c}
E_1 & E_2 \\
\hline
E_3 & E_4
\end{array}
\right] \in \mathbbm{R}^{2n \times 2n},
\]
\[
F=
\left[
\begin{array}{c|c}
F_1 & F_2 \\
\hline
F_3 & F_4
\end{array}
\right] \in \mathbbm{R}^{2n \times 2n},
\]
\[
G=
\left[
\begin{array}{c|c}
G_1 & G_2 \\
\hline
G_3 & G_4
\end{array}
\right] \in \mathbbm{R}^{2n \times 2n}.
\]
\\

\noindent \textbf{Lemma 1.}  If $G_1 = G_4 = E_1 = E_4  = 0$ and $G_2^T + G_3$, $E_2^T + E_3$ are 0, (in other words, E, G are \textit{anti-symmetric} matrix). Then the \textit{non-canonical} Hamiltonian dynamics preserves $H$.
\\
\\
\textit{proof.} The non-canonical hamiltonian dynamics is following: 
\[
\frac{d}{dt}
\left[
\begin{array}{ccc}
q_1 \\
q_2 \\
p_1 \\
p_2 
\end{array}
\right]
=
\left[
\begin{array}{cclcc}
E_{1} & E_{2} & F_{1} & F_{2} \\
E_3 & E_4 & F_3 & F_4 \\
-F_1^T & -F_3^T & G_1 & G_2 \\
-F_2^T & -F_4^T & G_3 & G_4 
\end{array}
\right]
\left[
\begin{array}{ccc}
\nabla_{q_1} U(q_1) \\
\nabla_{q_2} U(q_2) \\
p_1 \\
p_2 
\end{array}
\right]
\] 
Since $G_1 = G_4 = E_1 = E_4  = 0$,
\[
\left[
\begin{array}{cclcc}
E_{1} & E_{2} & F_{1} & F_{2} \\
E_3 & E_4 & F_3 & F_4 \\
-F_1^T & -F_3^T & G_1 & G_2 \\
-F_2^T & -F_4^T & G_3 & G_4 
\end{array}
\right]
\left[
\begin{array}{ccc}
\nabla_{q_1} U(q_1) \\
\nabla_{q_2} U(q_2) \\
p_1 \\
p_2 
\end{array}
\right] = 
\left[
\begin{array}{ccc}
E_2 \nabla_{q_2} U(q_2) + F_1 p_1 + F_2 p_2 \\
E_3 \nabla_{q_1} U(q_1) + F_3 p_1 + F_4 p_4 \\
-F_1^T \nabla_{q_1} U(q_1) - F_3^T \nabla_{q_2} U(q_2)  + G_2 p_2\\
-F_2^T \nabla_{q_1} U(q_1) - F_4^T \nabla_{q_2} U(q_2)  + G_3 p_1\\
\end{array}
\right]
\] 
\\
\\
Also, note that the poisson bracket is 

$$\displaystyle\frac{dQ}{dt} = \sum_{i=1}^{2n} \frac{\partial Q} {\partial p_i} \frac{dp_i}{dt}+  \sum_{i=1}^{2n} \frac{\partial Q} {\partial q_i} \frac{dq_i}{dt}$$

\noindent Therefore, putting previous evaluations, 
$$\displaystyle\frac{dQ}{dt}
 = (\nabla_{p_1}Q)^T (-F_1^T \nabla_{q_1} U(q_1) - F_3^T \nabla_{q_2} U(q_2)  + G_2 p_2) 
+  (\nabla_{p_2}Q)^T(-F_2^T \nabla_{q_1} U(q_1) - F_4^T \nabla_{q_2} U(q_2) $$ $$ + G_3 p_1) 
+ (\nabla_{q_1}Q^T)(E_2 \nabla_{q_2} U(q_2) + F_1 p_1 + F_2 p_2) + (\nabla_{q_2}Q^T)(E_3 \nabla_{q_1} U(q_1) + F_3 p_1 + F_4 p_4)$$

This can be re-expressed as

$$\displaystyle\frac{dQ}{dt} = {(\nabla_{p_1}Q)^T}(-F_1^T \nabla_{q_1} U(q_1) - F_3^T\nabla_{q_2} U(q_2) ) +  
(\nabla_{p_2}Q)^T(-F_2^T \nabla_{q_1} U(q_1) - F_4^T \nabla_{q_2} U(q_2)) $$
$$+(\nabla_{q_1}Q^T)( F_1 p_1 + F_2 p_2) +
(\nabla_{q_2}Q^T)(F_3 p_1 + F_4 p_4) +
(\nabla_{p_1}Q)^TG_2 p_2+ 
(\nabla_{p_2}Q)^TG_3 p_1$$
$$+(\nabla_{q_1}Q^T)E_2 \nabla_{q_2} U(q_2) +(\nabla_{q_2}Q^T)E_3 \nabla_{q_1} U(q_1)$$.

Now, letting $Q=H$, we know 

$${(\nabla_{p_1}H)^T}(-F_1^T \nabla_{q_1} U(q_1) - F_3^T\nabla_{q_2} U(q_2) ) +  (\nabla_{p_2}H)^T(-F_2^T \nabla_{q_1} U(q_1) - F_4^T \nabla_{q_2} U(q_2)) $$
$$+  
(\nabla_{q_1}H^T)( F_1 p_1 + F_2 p_2) +
(\nabla_{q_2}H^T)(F_3 p_1 + F_4 p_4)=0$$.

Therefore, it is enough to find the condition for 

$$(\nabla_{p_1}H)^TG_2 p_2+ 
(\nabla_{p_2}H)^TG_3 p_1+
(\nabla_{q_1}H^T)E_2 \nabla_{q_2} U(q_2) +
(\nabla_{q_2}H^T)E_3 \nabla_{q_1} U(q_1) = 0$$.

By assumption of $H$, this is equivalent to 

$${p_1}^TG_2 p_2+ 
{p_2}^TG_3 p_1+
(\nabla_{q_1}U(q_1))^TE_2 \nabla_{q_2} U(q_2) +
(\nabla_{q_2}U(q_2))^TE_3 \nabla_{q_1} U(q_1) $$
$$={p_2}^T(G_2^T + G_3) p_1 + (\nabla_{q_2}U(q_2))^T(E_2^T + E_3) \nabla_{q_1} U(q_1) = 0$$

Since $G_2^T + G_3$ and $E_2^T + E_3 $ are zero, we completes the proof. 
\\
\\
\noindent \textbf{Lemma 2.}  We can make \textit{leapfrog like integrator} in this dynamic system.
\\
\\
\textit{proof.} We split our Hamiltonian into sub-Hamiltonians as follow :

$$\displaystyle H(q_1, q_2, p_1, p_2) = \frac{1}{2}U(q_1) + \frac{1}{2}U(q_2)+(\frac{1}{2}{p_1}^T p_1 + \frac{1}{2}{p_2}^T p_2 )+ \frac{1}{2}U(q_2) + \frac{1}{2}U(q_1) $$, 

with $$\displaystyle H_1 = \frac{1}{2}U(q_1)$$ $$\displaystyle H_2 = \frac{1}{2}U(q_2)$$ $$\displaystyle H_3 = \frac{1}{2}{p_1}^T p_1 + \frac{1}{2}{p_2}^T p_2 $$

Then, the corresponding non-canonical dynamics for each Hamiltonian are following :

\[
H_1 : \frac{d}{dt}
\left[
\begin{array}{ccc}
q_1 \\
q_2 \\
p_1 \\
p_2 
\end{array}
\right]
=
\left[
\begin{array}{cclcc}
0 & E_{2} & F_{1} & F_{2} \\
E_3 & 0& F_3 & F_4 \\
-F_1^T & -F_3^T & 0 & G_2 \\
-F_2^T & -F_4^T & G_3 & 0 
\end{array}
\right]
\left[
\begin{array}{ccc}
\nabla_{q_1} U(q_1) \\
0 \\
0 \\
0 
\end{array}
\right]
=
\left[
\begin{array}{ccc}
0 \\
E_3 \nabla_{q_1} U(q_1) \\
-F_1^T \nabla_{q_1} U(q_1) \\
-F_2^T \nabla_{q_1} U(q_1) 
\end{array}
\right]
\] 

\[
H_2 : \frac{d}{dt}
\left[
\begin{array}{ccc}
q_1 \\
q_2 \\
p_1 \\
p_2 
\end{array}
\right]
=
\left[
\begin{array}{cclcc}
0 & E_{2} & F_{1} & F_{2} \\
E_3 & 0& F_3 & F_4 \\
-F_1^T & -F_3^T & 0 & G_2 \\
-F_2^T & -F_4^T & G_3 & 0 
\end{array}
\right]
\left[
\begin{array}{ccc}
0\\
\nabla_{q_2} U(q_2)  \\
0 \\
0 
\end{array}
\right]
=
\left[
\begin{array}{ccc}
E_2 \nabla_{q_2} U(q_2)  \\
0 \\
-F_3^T\nabla_{q_2} U(q_2) \\
-F_4^T \nabla_{q_2} U(q_2) 
\end{array}
\right]
\] 

\[
H_3 : \frac{d}{dt}
\left[
\begin{array}{ccc}
q_1 \\
q_2 \\
p_1 \\
p_2 
\end{array}
\right]
=
\left[
\begin{array}{cclcc}
0 & E_{2} & F_{1} & F_{2} \\
E_3 & 0& F_3 & F_4 \\
-F_1^T & -F_3^T & 0 & G_2 \\
-F_2^T & -F_4^T & G_3 & 0 
\end{array}
\right]
\left[
\begin{array}{ccc}
0\\
0  \\
p_1 \\
p_2
\end{array}
\right]
=
\left[
\begin{array}{ccc}
F_1 p_1 + F_2 p_2  \\
F_3 p_1 + F_4 p_2  \\
G_2 p_2 \\
G_3 p_1 
\end{array}
\right]
\] 

which are all tractable. Therefore, for given $\epsilon > 0$ and $L$, we can construct a leapfrog operator for $H$

$$\displaystyle\Phi_{H,\epsilon} = \Phi_{H_1,\epsilon} \circ \Phi_{H_2,\epsilon} \circ \Phi_{H_3,\epsilon} \circ \Phi_{H_2,\epsilon} \circ \Phi_{H_1,\epsilon}$$


\noindent \textbf{Theorem 1.} There exists an operator for the probabilistic sampling operator in non-canonical Hamiltonian dynamics with two particles interacts each other.   
\\
\\
\textit{proof.} Note that $E$, $G$ and $p$ must be flipped for each loop in non-canonical Hamiltonian dynamics for time reversibility. Therefore, by adding these operators and the operator earned in \textbf{Lemma 2}, we earn the following volume-preserving, self-inverse, time-reversible operator for our algorithm. The operator is given as follows:

$$\displaystyle\tilde{\Phi}_{H,\epsilon} = \Phi_{p} \circ \Phi_{E,G} \circ (\Phi_{H_1,\epsilon} \circ \Phi_{H_2,\epsilon} \circ \Phi_{H_3,\epsilon} \circ \Phi_{H_2,\epsilon} \circ \Phi_{H_1,\epsilon})^L$$

Therefore, we can make an algorithm for sampling by making a loop with flipping block matrices and momentum (sampled by standard normal), and doing leapfrog steps with given step size $\epsilon$ (local) and $L$ (global).

\subsection{Algorithm}
\newcounter{rownumbers}
\newcommand\rownumber{\stepcounter{rownumbers}\arabic{rownumbers}}
\DeclarePairedDelimiter\floor{\lfloor}{\rfloor}
We established our algorithm based on the Magnetic Hamiltonian Monte Carlo algorithm.
\\
\\
\\
\\
\\
\\
\begin{algorithm}
\caption{Non-Canonical Hamiltonian Monte Carlo Algorithm for Two Particles}
\begin{algorithmic}
\STATE\begin{enumerate}
  \STATE Initialize $(q_1, q_2, p_1, p_2)_0$ on phase space.
  \STATE For n = 1, 2, \ldots, $\floor{\frac{T}{\epsilon}}$ where T is total time and $\epsilon$ is a discretized time quantum. 
  \begin{enumerate}
\item $(q_1, q'_2, p'_1, p'_2)_{n-1} \leftarrow \Phi_{H_1,\epsilon}(q_1, q_2, p_1, p_2)_{n-1}$
\item $(q'_1, q'_2, p''_1, p''_2)_{n-1} \leftarrow \Phi_{H_2,\epsilon}(q_1, q'_2, p'_1, p'_2)_{n-1}$
\item $(q''_1, q''_2, p'''_1, p'''_2)_{n-1} \leftarrow \Phi_{H_3,\epsilon}(q'_1, q'_2, p''_1, p''_2)_{n-1}$
\item $(q''_1, q'''_2, p''''_1, p''''_2)_{n-1} \leftarrow \Phi_{H_2,\epsilon}(q''_1, q''_2, p'''_1, p'''_2)_{n-1}$
\item $(q'''_1, q'''_2, p'''''_1, p'''''_2)_{n-1} \leftarrow \Phi_{H_1,\epsilon}(q''_1, q'''_2, p''''_1, p''''_2)_{n-1}$
\item If $$ Unif([0, 1])<min(1,exp(H((q_1, q_2, p_1, p_2)_{n-1})
-H((q'''_1, q'''_2, p'''''_1, p'''''_2)_{n-1})$$ \\
then $(q_1, q_2, p_1, p_2)_{n} \leftarrow (q'''_1, q'''_2, p'''''_1, p'''''_2)_{n-1}$ \\
otherwise $(q_1, q_2, p_1, p_2)_{n} \leftarrow (q_1, q_2, p_1, p_2)_{n-1}$
\end{enumerate}
\STATE Return $(q_1, q_2)_{n}$ 
\end{enumerate}

\end{algorithmic}
\end{algorithm}

\section{Conclusion}

Probability density sampling algorithms are very important for probabilistic inference, since it is hard to evaluate the exact value and functions with closed form in most cases. However, pre-existing probabilistic density sampling algorithms for probabilistic inferences still have problems in multimodal distributions or high-dimensional cases. Therefore, in this research, we tried to apply many existing physical theories such as Noether's Theorem, non-canonical Hamiltonian dynamics to propose better probabilistic inference algorithm for these situations. 

Applying non-canonical Hamiltonian dynamics with independent probability distribution and two particles, we proposed two particle sampling algorithm for density sampling interacting each other. We copied and pasted the target distribution that we want to sample for making the assumption for independent distribution, and used non-canonical Hamiltonian dynamics to make interaction between these two particles. By interacting each other via non-canonical Hamiltonian dynamics, particles can be sampled in more various situation and get various traces. Setting proper parameters for two particles' interaction by real implementation and experiments, we expect more efficient, broad probabilistic density sampling will be available.  

\section{Further Research}

We theoretically verified sampling algorithms for multiple particles with interaction is possible. Yet, we don't know this algorithm gives better results via interaction in real. Therefore, based on the real implementation of our results, setting various parameters and distribution, we need to check which interaction made by certain parameters could make the sampling more efficiently. 

Furthermore, our algorithm can be generalized with more than 2 particles, not just 2 particles. If our algorithm is efficient with multimodal distribution sampling, we can increase more particles by the number of nodes in distributions.  By balancing the efficiency earned by interaction and increased computation time via increasing the particle, we can optimize the number for sampling for given distribution to sample.

\section*{References}


\small

[1] S. Duane, A.D. Kennedy, B. J. Pendleton, D. Roweth, (1987) Hybrid Monte Carlo, Physics Letters B.

[2] S.P. Meyn \ \& R.L Tweedie\ (2005) Markov Chains and Stochastic Stability, Springer-Verlag.

[3] Charles J. Geyer D.\ (2005) Markov Chain Monte Carlo Lecture Notes

[4] Michael Betancourt \ (2014) A Conceptual Introduction Hamiltonian Monte Carlo 

[5] Radford M. Neal \ (2011) MCMC using Hamiltonian dynamics, Appears as Chapter 5 of the Handbook of Markov Chain Monte Carlo, CRC Press, 2011

[6] Paul Vanetti, George Deligiannidis \ (2018) Piecewise-Deterministic Markov Chain Monte Carlo

[7] Nilesh Tripuraneni, Mark Rowland, Zoubin Ghahramani, Richard Turner\ (2017) Magnetic Hamiltonian Monte Carlo
\end{document}